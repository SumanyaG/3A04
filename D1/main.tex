\documentclass[]{article}
\usepackage{hyperref}
% Imported Packages
%------------------------------------------------------------------------------
\usepackage{amssymb}
\usepackage{amstext}
\usepackage{amsthm}
\usepackage{amsmath}
\usepackage{enumerate}
\usepackage{fancyhdr}
\usepackage[margin=1in]{geometry}
\usepackage{graphicx}
%\usepackage{extarrows}
%\usepackage{setspace}
%\usepackage{xcolor}
\usepackage{color}
%------------------------------------------------------------------------------

% Header and Footer
%------------------------------------------------------------------------------
\pagestyle{plain}  
\renewcommand\headrulewidth{0.4pt}                                      
\renewcommand\footrulewidth{0.4pt}                                    
%------------------------------------------------------------------------------

% Title Details
%------------------------------------------------------------------------------
\title{Deliverable \#1 Template : Software Requirement Specification (SRS)}
\author{SE 3A04: Software Design II -- Large System Design}
\date{18 February 2024}
                            
%------------------------------------------------------------------------------

% Document
%------------------------------------------------------------------------------
\begin{document}

\maketitle	
\noindent{\bf Tutorial Number:} T01\\
{\bf Group Number:} G07 \\
{\bf Group Members:} 
\begin{itemize}
	\item Awurama Nyarko
	\item Chelsea Maramot
 	\item Harrison Chiu
  	\item Khushi Bhojane
   	\item Sumanya Gulati
\end{itemize}

\section*{IMPORTANT NOTES}
\begin{itemize}
	\item Be sure to include all sections of the template in your document regardless whether you have something to write for each or not
	\begin{itemize}
		\item If you do not have anything to write in a section, indicate this by the \emph{N/A}, \emph{void}, \emph{none}, etc.
	\end{itemize}
	\item Uniquely number each of your requirements for easy identification and cross-referencing
	\item Highlight terms that are defined in Section~1.3 (\textbf{Definitions, Acronyms, and Abbreviations}) with \textbf{bold}, \emph{italic} or \underline{underline}
	\item For Deliverable 1, please highlight, in some fashion, all (you may have more than one) creative and innovative features. Your creative and innovative features will generally be described in Section~2.2 (\textbf{Product Functions}), but it will depend on the type of creative or innovative features you are including.
\end{itemize}

\newpage
\section{Introduction}
\label{sec:introduction}
% Begin Section

\begin{itemize}
	\item Provide an overview of the document/SRS.
\end{itemize}


\subsection{Purpose}
\label{sub:purpose}
% Begin SubSection
\begin{itemize}
	\item Specify the purpose of the SRS.
	\item Specify the intended audience for the SRS.
\end{itemize}
% End SubSection

\subsection{Scope}
\label{sub:scope}
% Begin SubSection
\begin{itemize}
	\item Identify the software product(s) to be produced, and name each (e.g., Host DBMS, Report Generator, etc.)
	\item Explain what the software product(s) will do (and, if necessary, also state what they will not do).
	\item Describe the application of the software being specified, including relevant benefits, objectives, and goals.
%	\item Be consistent with similar statements in higher-level specifications (e.g., the system requirements specification), if they exist
\end{itemize}
% End SubSection

\subsection{Definitions, Acronyms, and Abbreviations}
\label{sub:definitions_acronyms_and_abbreviations}
% Begin SubSection
\begin{itemize}
	\item Provide the definitions of all terms, acronyms, and abbreviations required to properly interpret the SRS.
	\item This should be in alphabetical order.
\end{itemize}
% End SubSection

\subsection{References}
\label{sub:references}
% Begin SubSection
\begin{itemize}
	\item Provide a complete list of all documents referenced elsewhere in the SRS.
	\item Identify each document by title, report number (if applicable), date, and publishing organization.
	\item Specify the sources from which the references can be obtained.
	\item Order this list in some sensible manner (alphabetical by author, or something else that makes more sense).
	\begin{thebibliography}{99}
		\bibitem{1c} “5 Ultimate Steps To Customize Your Windows Folders to Increase Productivity.” Accessed: Feb. 17, 2024. [Online]. Available: \url{https://www.linkedin.com/pulse/5-ultimate-steps-customize-your-windows-folders-increase-do/}
		
		\bibitem{2c} “Web Content Accessibility Guidelines (WCAG) 2.1.” Accessed: Feb. 17, 2024. [Online]. Available: \url{https://www.w3.org/TR/WCAG21/}
		
		\bibitem{3c} “How To Set Up Optimal Chat for Your Real-Time Application.” Accessed: Feb. 17, 2024. [Online]. Available: \url{https://subspace.com/resources/optimal-chat-real-time}
		
		\bibitem{4c} N. Mansour, “Mobile App Quality: An Essential Guide | Instabug.” Accessed: Feb. 17, 2024. [Online]. Available: \url{https://www.instabug.com/blog/mobile-app-quality-an-essential-guide}
		
		\bibitem{5c} “8 Ways to Effectively Reduce Server Response Time,” DataDome. Accessed: Feb. 17, 2024. [Online]. Available: \url{https://datadome.co/learning-center/how-to-reduce-server-response-time/}
		
		\bibitem{6c} SentinelOne, “SentinelOne | Service Availability: What It Is and Metrics You Should Know,” SentinelOne. Accessed: Feb. 17, 2024. [Online]. Available: \url{https://www.sentinelone.com/blog/service-availability-and-metrics/}
		
		\bibitem{7c} “Our Experience with Modular Architecture,” TRIARE. Accessed: Feb. 17, 2024. [Online]. Available: \url{https://triare.net/insights/modular-architecture-2/}
		
		\bibitem{8c} kavinrajan, “Suggested Minimum Android version 2021,” Medium. Accessed: Feb. 17, 2024. [Online]. Available: \url{https://medium.com/@kavinece53/suggested-minimum-android-version-2021-affba3ef74f}
		
		\bibitem{9c} “Google Play.” Accessed: Feb. 17, 2024. [Online]. Available: \url{https://play.google.com/about/developer-distribution-agreement.html}
		
		\bibitem{10c} O. of the P. C. of Canada, “PIPEDA fair information principles.” Accessed: Feb. 17, 2024. [Online]. Available: \url{https://www.priv.gc.ca/en/privacy-topics/privacy-laws-in-canada/the-personal-information-protection-and-electronic-documents-act-pipeda/p_principle/}
		
		\bibitem{11c} “Core app quality,” Android Developers. Accessed: Feb. 17, 2024. [Online]. Available: \url{https://developer.android.com/docs/quality-guidelines/core-app-quality}
	\end{thebibliography}
\end{itemize}
% End SubSection

\subsection{Overview}
\label{sub:overview}
% Begin SubSection
\begin{itemize}
	\item Describe what the remainder of the document/SRS contains.\\
	(e.g. "Section 2 discusses...Section 3...")
%	\item Explain how the SRS is organized
\end{itemize}
% End SubSection

% End Section

\section{Overall Product Description}
\label{sec:overall_description}
% Begin Section

\begin{itemize}
	\item This section should describe the general factors that affect the product and its requirements. 
	\item It does not state specific requirements.
	\item It provides a \emph{background} for those requirements and makes them easier to understand.
\end{itemize}


\subsection{Product Perspective}
\label{sub:product_perspective}
% Begin SubSection
\begin{itemize}
	\item Put the product into perspective with other related products, i.e., context
	\item If the product is independent and totally self-contained, it should be stated here
	\item If the SRS defines a product that is a component of a larger system, then this subsection should relate the requirements of that larger system to the functionality of the software being developed. Identify interfaces between that larger system and the software to be developed.
	\item A block diagram showing the major components of the larger system, interconnections, and external interfaces can be helpful
\end{itemize}
% End SubSection

\subsection{Product Functions}
\label{sub:product_functions}
% Begin SubSection
\begin{itemize}
	\item Provide a \emph{summary} of the major functions that the software will perform.
	\begin{itemize}
		\item \textbf{Example}: An SRS for an accounting program may use this part to address customer account maintenance, customer statement, and invoice preparation without mentioning the vast amount of detail that each of those functions requires.
	\end{itemize}
	\item Functions should be organized in a way that makes the list of functions understandable to the customer or to anyone else reading the document for the first time 
	\item Present the functions in a list format - each item should be one function, with a brief description of it
	\item Textual or graphical methods can be used to show the different functions and their relationships
	\begin{itemize}
		\item Such a diagram is not intended to show a design of a product, but simply shows the logical relationships among variables
	\end{itemize} 
\end{itemize}
% End SubSection

\subsection{User Characteristics}
\label{sub:user_characteristics}
% Begin SubSection
\begin{itemize}
	\item Describe those general characteristics of the intended users of the product including educational level, experience, and technical expertise 
	\item Since there will be many users, you may wish to divide into different user types or personas
%	\item Do not state specific requirements, but rather provide the reasons why certain specific requirements are later specified
\end{itemize}
% End SubSection

\subsection{Constraints}
\label{sub:constraints}
% Begin SubSection
\begin{itemize}
	\item Provide a general description of any constraints that will limit the developer's options
\end{itemize}
% End SubSection

\subsection{Assumptions and Dependencies}
\label{sub:assumptions_and_dependencies}
% Begin SubSection
\begin{itemize}
	\item List any assumptions you made in interpreting what the software being developed is aiming to achieve
	\item List any other assumptions you made that, if it fails to hold, could require you to change the requirements
	%\item List each of the factors that affect the requirements stated in the SRS
	%\item These factors are not design constraints on the software but are, rather, any changes to them that can affect the requirements in the SRS
	\begin{itemize}
		\item \textbf{Example}: An assumption may be that a specific operating system will be available on the hardware designated for the software product. If, in fact, the operating system is not available, the SRS would then have to change accordingly.
	\end{itemize}
\end{itemize}
% End SubSection

\subsection{Apportioning of Requirements}
\label{sub:apportioning_of_requirements}
% Begin SubSection
\begin{itemize}
	\item Identify requirements that may be delayed until future versions of the system
\end{itemize}
% End SubSection

% End Section
\section{Use Case Diagram}
\label{sec:use_case_diagram}
% Begin Section
\begin{itemize}
	\item Provide the use case diagram for the system being developed.
	\item You do not need to provide the textual description of any of the use cases here (these will be specified under "Highlights of Functional Requirements").
%	\item Provide \emph{one} use case diagram for the most important Business Event.
%	\item The text of all use cases will be specified under "Highlights of Functional Requirements"
\end{itemize}
%In this section, select the most important Business Event that your system responds to and give its use case diagram.  Only one use case diagram is needed.  Give a brief textual description of the use case without repeating what is in the scenarios of the corresponding Business Event.

%
%
%
%This section should provide a use case diagram for your application. 
%\begin{enumerate}[a)]
%	\item Each use case appearing in the diagram should be accompanied by a text description. 
%\end{enumerate}
%% End Section

\section{Highlights of Functional Requirements}
\label{sec:functional_requirements}
% Begin Section
\begin{itemize}
	\item Specify all use cases (or other scenarios triggered by other events), organized by Business Event. 
	\item For each Business Event, show the scenario from every Viewpoint. You should have the same set of Viewpoints across all Business Events. If a Viewpoint doesn't participate, write N/A so we know you considered it still. You can choose how to present this - keep in mind it should be easy to follow. 
	\item At the end, combine them all into a Global Scenario.
	%\item Specify the "use cases" (or other triggering events) organized by Business Event. (The Global Scenario is what you might think of as a use case). Be sure to consider Business Events that aren't just triggered by users with goals (e.g. something happens in the environment that your system needs to respond to)
	\item Your focus should be on what the system needs to do, not how to do it. Specify it in enough detail that it clearly specifies what needs to be accomplished, but not so detailed that you start programming or making design decisions.
	\item Keep the length of each use case (Global Scenario) manageable. If it's getting too long, split into sub-cases.
	\item You are \emph{not} specifying a complete and consistent set of functional requirements here. (i.e. you are providing them in the form of use cases/global scenarios, not a refined list). For the purpose of this project, you do not need to reduce them to a list; the global scenarios format is all you need.
	\item Red text below is just to highlight where you need to insert a scenario - don't actually write it all in red.
\end{itemize}

\noindent {\bf Main Business Events:} List out all the main business events you are presenting. If you sub-divided into smaller ones, you don't need to include the smaller ones in this list.\\

\noindent {\bf Viewpoints:} List out all the viewpoints you will be considering.\\

\noindent {\bf Interpretation:} Specify any liberties you took in interpreting business events, if necessary.\\

\begin{enumerate}[{\bf BE1.}]
	\item Authorizing an employee to use the chat application for work purposes. \#1
		\begin{enumerate}[{\bf VP1.}]
			\item Employee user \#1 \\
				\textcolor{black}{ \bf Main Success Scenario
    				1.System displays required login fields.
				2.User inputs personal login information.
				3.Information is verified by the system.
				4.User is sent a verification code.
				5.User enters code.
				6.System reviews code and if correct, allows user access to the application.
    				\bf Secondary Scenario
				3i. The information entered is incorrect.
	      				3.i.1. The user enters incorrect information.
	     				3.i.2. The system sends an error message.
	      				3.i.3. Loop back to BE2.VP1.1
				4i. The user does not receive the code.
	      				4.i.1. The user asks the system to resend the code.
	      				4.i.2. The system sends a new code.
	      				4.i.3. Loop back to BE2.VP1.4
     				5i. The code entered is incorrect.
	      				5.i.1. The user enters incorrect code.
	      				5.i.2. The system sends an error message.
	      				5.i.3. Loop back to BE2.VP1.5


}
			\item Organization \#2 \\
				\textcolor{red}{N/A}
    			\item Board of Directors \#3 \\
				\textcolor{red}{N/A}
			\item IT System Support \#4 \\
				\textcolor{red}{N/A}
		\end{enumerate}
		{\bf Global Scenario:}\\
		\textcolor{red}{Insert Scenario Here}
	\item Business Event Name \#2
	\begin{enumerate}[{\bf VP1.}]
		\item Viewpoint Name \#1 \\
		\textcolor{red}{Insert Scenario Here}
		\item Viewpoint Name \#2 \\
		\textcolor{red}{Insert Scenario Here}
	\end{enumerate}
	{\bf Global Scenario:}\\
	\textcolor{red}{Insert Scenario Here}
\end{enumerate}

%	Below, we organize by Business Event.
%	\begin{enumerate}[{BE}1.]
%		\item Business Event name
%		\begin{enumerate}[{VP1}.1]
%			\item Viewpoint name \newline
%			\noindent\fbox{%
%				\parbox{0.5\textwidth}{%
%					\begin{itemize}
%						\item {\bf $S_{1}$:} Initial response of the system to the Business Event
%						\item {\bf $E_{1}$:}  Reaction of the environment to $S_{1}$
%						\item {\bf $S_{2}$:}  Response of the system to $E_{1}$
%						\item {\bf $E_{2}$:}  Reaction of the environment to $S_{2}$
%						\item[] $\cdots$
%						\item {\bf $S_{n}$:}  Response of the system to $E_{(n-1)}$
%						\item {\bf $E_{n}$:}  Reaction of the environment to $E_{(n-1)}$
%						\item {\bf $S_{(n+1)}$:} Final response of the system concluding its function regarding the Business Event
%					\end{itemize}
%				}%
%			}
%			\item Viewpoint name\newline
%			\noindent\fbox{%
%				\parbox{0.5\textwidth}{%
%					\begin{itemize}
%						\item {\bf $S_{1}$:} Initial response of the system to the Business Event
%						\item {\bf $E_{1}$:}  Reaction of the environment to $S_{1}$
%						\item {\bf $S_{2}$:}  Response of the system to $E_{1}$
%						\item {\bf $E_{2}$:}  Reaction of the environment to $S_{2}$
%						\item[] $\cdots$
%						\item {\bf $S_{k}$:}  Response of the system to $E_{(k-1)}$
%						\item {\bf $E_{k}$:}  Reaction of the environment to $E_{(k-1)}$
%						\item {\bf $S_{(k+1)}$:} Final response of the system concluding its function regarding the Business Event
%					\end{itemize}
%				}%
%			}
%			\item \dots
%			\item \dots
%			\item \dots
%			\item[\dots]
%		\end{enumerate}	
%		\item[] {\bf Global Scenario of {\it Business Event Name}:} It is the scenario corresponding to the integration of all the above scenarios from the different Viewpoints of the Business Event BE1.\newline
%		\noindent\fbox{%
%			\parbox{0.5\textwidth}{%
%				\begin{itemize}
%					\item {\bf $S_{1}$:} Initial response of the system to the Business Event
%					\item {\bf $E_{1}$:}  Reaction of the environment to $S_{1}$
%					\item {\bf $S_{2}$:}  Response of the system to $E_{1}$
%					\item {\bf $E_{2}$:}  Reaction of the environment to $S_{2}$
%					\item[] $\cdots$
%					\item {\bf $S_{m}$:}  Response of the system to $E_{(m-1)}$
%					\item {\bf $E_{m}$:}  Reaction of the environment to $E_{(m-1)}$
%					\item {\bf $S_{(m+1)}$:} Final response of the system concluding its function regarding the Business Event
%				\end{itemize}
%			}%
%		}	
%		%\end{enumerate}
%		\item Business Event name
%		\begin{enumerate}[{VP1}.1]
%			\item Viewpoint name \newline
%			\noindent\fbox{%
%				\parbox{0.5\textwidth}{%
%					\begin{itemize}
%						\item {\bf $S_{1}$:} Initial response of the system to the Business Event
%						\item {\bf $E_{1}$:}  Reaction of the environment to $S_{1}$
%						\item {\bf $S_{2}$:}  Response of the system to $E_{1}$
%						\item {\bf $E_{2}$:}  Reaction of the environment to $S_{2}$
%						\item[] $\cdots$
%						\item {\bf $S_{n'}$:}  Response of the system to $E_{(n'-1)}$
%						\item {\bf $E_{n'}$:}  Reaction of the environment to $E_{(n'-1)}$
%						\item {\bf $S_{(n'+1)}$:} Final response of the system concluding its function regarding the Business Event
%					\end{itemize}
%				}%
%			}
%			\item Viewpoint name\newline
%			\noindent\fbox{%
%				\parbox{0.5\textwidth}{%
%					\begin{itemize}
%						\item {\bf $S_{1}$:} Initial response of the system to the Business Event
%						\item {\bf $E_{1}$:}  Reaction of the environment to $S_{1}$
%						\item {\bf $S_{2}$:}  Response of the system to $E_{1}$
%						\item {\bf $E_{2}$:}  Reaction of the environment to $S_{2}$
%						\item[] $\cdots$
%						\item {\bf $S_{k'}$:}  Response of the system to $E_{(k'-1)}$
%						\item {\bf $E_{k'}$:}  Reaction of the environment to $E_{(k'-1)}$
%						\item {\bf $S_{(k'+1)}$:} Final response of the system concluding its function regarding the Business Event
%					\end{itemize}
%				}%
%			}
%			\item \dots
%			\item \dots
%			\item \dots
%			\item[\dots]
%		\end{enumerate}	
%		\item[] {\bf Global Scenario of {\it Business Event Name}:} It is the scenario corresponding to the integration of all the above scenarios from the different Viewpoints of the Business Event BE2.\newline
%		\noindent\fbox{%
%			\parbox{0.5\textwidth}{%
%				\begin{itemize}
%					\item {\bf $S_{1}$:} Initial response of the system to the Business Event
%					\item {\bf $E_{1}$:}  Reaction of the environment to $S_{1}$
%					\item {\bf $S_{2}$:}  Response of the system to $E_{1}$
%					\item {\bf $E_{2}$:}  Reaction of the environment to $S_{2}$
%					\item[] $\cdots$
%					\item {\bf $S_{m'}$:}  Response of the system to $E_{(m'-1)}$
%					\item {\bf $E_{m'}$:}  Reaction of the environment to $E_{(m'-1)}$
%					\item {\bf $S_{(m'+1)}$:} Final response of the system concluding its function regarding the Business Event
%				\end{itemize}
%			}%
%		}		
%	\end{enumerate}

%End Section

\section{Non-Functional Requirements}
\label{sec:non-functional_requirements}


\begin{itemize}
	\item For each non-functional requirement, provide a justification/rationale for it.\\
	{\bf Example:} \\
	SC1. \emph{The device should not explode in a customer’s pocket.}\\
	{\bf Rationale:} Other companies have had issues with the batteries they used in their phones randomly exploding [insert citation]. This causes a safety issue, as the phone is often carried in a person's hand or pocket.	
	\item If you need to make a guess because you couldn't really talk to stakeholders, you can say "We imagined stakeholders would want...because..."
	\item Each requirement should have a unique label/number for it.
	\item In the list below, if a particular section doesn't apply, just write N/A so we know you considered it.
\end{itemize}





% Begin Section
\subsection{Look and Feel Requirements}
\label{sub:look_and_feel_requirements}   
% Begin SubSection

\subsubsection{Appearance Requirements}
\label{ssub:appearance_requirements}
% Begin SubSubSection
\begin{enumerate}[{LF-A}1. ]
	\item The system colors shall adhere to the organization's branding. \\
	{\bf Rationale:} Utilization of company colors is crucial for maintaining a unified professional image, allowing the application to be 
	recognizable to its users.
	\item The system shall use clear and recognizable icons and images. \\
	{\bf Rationale:} Clear and recognizable icons and images will improve application usability, helping users recognize the meaning behind
	the icons and images. To be clear and recognizable, a standard icon should be high quality and have excellent pixel resolution (24x24 pixels to 48x48 pixels)
	\cite{1c}.
	\item The system shall integrate the company seal or emblem within the application design to emphasize professional identity. \\
	{\bf Rationale:} The inclusion of the company seal reinforces brand authenticity and professionalism.
	\item Users shall be able to distinguish between incoming and outgoing messages. The system shall display messages in the chat application with distinguishable sender information, timestamp, and message content.  
	To achieve a distinguishable message format, the system shall use the traditional chat-bubble conversation layout of standard
	chat applications.\\
	{\bf Rationale:} A distinguishable message format makes it easier for the users to follow discussions and contributes to an organized and 
	user-friendly interface. 
	\item The system shall refrain from utilizing bright and saturated colors as background elements. \\
	{\bf Rationale:} The system will avoid the use of vibrant and intense hues that could distract and overwhelm users. This creates a 
	visually balanced and user-friendly interface.
\end{enumerate}
% End SubSubSection

\subsubsection{Style Requirements}
\label{ssub:style_requirements}
% Begin SubSubSection
\begin{enumerate}[{LF-S}1. ]
	\item The application shall adhere to a consistent color scheme throughout the user interface. \\
	{\bf Rationale:} A consistent color scheme provides a cohesive and unified design. A consistent colour scheme
	refers to the use of predefined colors that remains uniform across various components of the application.
	\item The system shall use a font that is legible and readable across different devices, screen resolutions, and lighting conditions. \\
	{\bf Rationale:} A readable and professional font would allow users to read messages, promoting effective communication among users. 
	\item The application shall have a clean and professional design reflecting the corporate environment. 
	{\bf Rationale:} A professional design will use a color palette, font, and visual elements that convey a sense
	of professionalism and reliability.
	\item The application shall support responsive design to adapt to various screen sizes and orientations. \\
	{\bf Rationale:} There are numerous screen sizes and thus, supporting responsive design would ensure that the application 
	maintains a consistent and visually appealing layout across various device sizes, contributing to a seamless user experience.
\end{enumerate}
% End SubSubSection

% End SubSection


\subsection{Usability and Humanity Requirements}
\label{sub:usability_and_humanity_requirements}
% Begin SubSection

\subsubsection{Ease of Use Requirements}
\label{ssub:ease_of_use_requirements}
% Begin SubSubSection
\begin{enumerate}[{UH-EOU}1. ]
	\item The system shall have a clear and intuitive navigation layout, ensuring that users can effortlessly locate 
	the information that they seek. An intuitive layout follows the traditional organization of existing chat applications. \\
	{\bf Rationale:} An intuitive design ensures that users can seamlessly engage with the application features with minimal learning curve. 
	This would enhance overall usability and contributes to increased user satisfaction. 
	\item The system shall allow users to quickly access recent chats, allowing users to efficiently engage with recent 
	conversations without unnecessary navigation. Recent chat conversations should be located at the top of the users list of conversations.\\
	{\bf Rationale:} Minimizing user effort to access recent chats ensures seamless user experience and promotes swift interaction 
	with relevant contacts. Recognizable icons with clear action indicator are contextually relevant and avoids metamorphic imagery. 
	\item The system shall use icons that clearly indicate the associated action or feature, giving users a visual cue of what each icon represents. \\
	{\bf Rationale:} This enhances the usability of the application, allowing users to navigate seamlessly through the application. 
	\item The system shall incorporate an in-app reporting tool which enables users to report bugs encountered in the application.\\
	{\bf Rationale:} This requirements allows users to inform developers of any bugs, prompting necessary fixes.
\end{enumerate} 

% End SubSubSection

\subsubsection{Personalization and Internationalization Requirements}
\label{ssub:personalization_and_internationalization_requirements}
% Begin SubSubSection
\begin{enumerate}[{UH-PI}1. ]
	\item The system shall allow users to customize their profiles through addition of personal information (i.e., name, age, gender, date of birth, hometown), profile pictures, and adjustment of chat availability status. \\
	{\bf Rationale:} Allowing the users to personalize their profiles would foster a sense of identity and self-expression within the application.
\end{enumerate}
% End SubSubSection


\subsubsection{Learning Requirements}
\label{ssub:learning_requirements}
% Begin SubSubSection
\begin{enumerate}[{UH-L}1. ]
	\item The system shall facilitate a user-friendly design that enables users to intuitively navigate and comprehend the chat application's key
	 features within a maximum time frame of 10 minutes.\\
	{\bf Rationale:} The absence of any tutorials necessitates achieving an intuitive navigation design
	that would enable the users to swiftly understand and utilize the chat application within a couple of minutes of using.
\end{enumerate}
% End SubSubSection

\subsubsection{Understandability and Politeness Requirements}
\label{ssub:understandability_and_politeness_requirements}
% Begin SubSubSection
\begin{enumerate}[{UH-UP}1. ]
	\item The system shall display informative and user-friendly error messages when an error is encountered or input is invalid. 
	An informative error message should consist of clear instructions on how to rectify the error written in natural language. \\
	{\bf Rationale:} A clear error message can help users to understand the nature of the problem. This allows the users to 
	take appropriate action to resolve the issue, reducing user frustration when an error is encountered.
	\item The system shall use symbols and icons that are culturally neutral and have universally positive associations and meaning.\\
	{\bf Rationale:} This requirement entails the avoidance of using symbols that may have different meanings or connotations
	in various cultures. This means opting for widely used and recognizable symbols.
\end{enumerate}
% End SubSubSection

\subsubsection{Accessibility Requirements}
\label{ssub:accessibility_requirements}
% Begin SubSubSection
\begin{enumerate}[{UH-A}1. ]
	\item The system shall adhere to Web Content Accessibility Guidelines (WCAG) accessibility standards to ensure usability among those with disabilities. This includes providing 
	alternative text for images and using sufficient color contrast between text and background elements. Text should have a color contrast ratio of at least 4.5:1 and larger text should
	have at least 3:1 contrast ratio \cite{2c}. \\
	{\bf Rationale:} Adhering to WCAG standards would ensure that the application is inclusive and accessible to a wider audience.
\end{enumerate} 

% End SubSubSection

% End SubSection




\subsection{Performance Requirements}
\label{sub:performance_requirements}
% Begin SubSection

\subsubsection{Speed and Latency Requirements}
\label{ssub:speed_and_latency_requirements}
% Begin SubSubSection
\begin{enumerate}[{PR-SL}1. ]
	\item The system shall deliver messages sent within the chat application to recipients in real-time, with a latency
	of no more than 250 milliseconds. \\
	{\bf Rationale:} A minimal latency will ensure that users experience near-instantaneous communication.
	A latency of 250 milliseconds is acceptable for real-time chat and other interactive messaging applications
	\cite{3c}.
	\item The system shall respond to user inputs within 2 seconds or less. \\
	{\bf Rationale:} A fast response time would enhance user experience making interactions within the application 
	seem immediate and responsive. A study has shown that 49\% of users expect an application to respond within 2 seconds
	\cite{4c}.
	\item The system shall allow users to query chat history with a response time of no more than 500 milliseconds. \\
	{\bf Rationale:} A swift search query response time enable users to efficiently locate messages in chat history, improving 
	usability of the application. A response time between 200 milliseconds and 1 second is considered 
	acceptable and will not hinder user experience
	\cite{5c}. 
	\item The system shall process and display response suggestions within a maximum of 500 milliseconds. \\
	{\bf Rationale:} This requirement provides users a seamless and responsive user experience, reducing 
	the latency of the suggestion feature. A response time between 200 milliseconds and 1 second is considered 
	acceptable and will not hinder user experience 
	\cite{5c}. 
\end{enumerate}
% End SubSubSection

\subsubsection{Safety-Critical Requirements}
\label{ssub:safety_critical_requirements}
% Begin SubSubSection
\begin{enumerate}[{PR-SC}1. ]
	\item The system shall implement a robust user authentication and authorization mechanism, ensuring only authorized
	personnel can access and participate in sensitive chat application communications. \\
	{\bf Rationale:} A robust user authentication reduces the risk of unauthorized access and protects sensitive information
	from being compromised.
	\item The system shall securely store a chat history log containing identifiers, date and time of communication, and an accurate chat log.\\
	{\bf Rationale:} Storing chat log is essential for auditing, accountability, and traceability. \\
\end{enumerate}
% End SubSubSection

\subsubsection{Precision or Accuracy Requirements}
\label{ssub:precision_or_accuracy_requirements}
% Begin SubSubSection
\begin{enumerate}[{PR-PA}1. ]
	\item The system shall maintain a timestamp accuracy of $\pm 2$ seconds, ensuring that the displayed
	time of sent and received messages accurately reflect the time they were processed. \\
	{\bf Rationale:} Ensuring high precision in the timestamp of message delivery contributes to the reliability
	and real-time nature of the chat application,  providing a responsive and communication experience to users.
	\item  The system shall update the user status indicator (online, offline, on vacation) within 1 second of the user 
	status changing. \\
	{\bf Rationale:} This requirement ensures that users have a precise understanding of the availability of other user for effective communication.
	A response time between 200 milliseconds and 1 second is considered acceptable and will not hinder user experience 
	\cite{5c}. 
	\item  The system shall accurately log 99.9\% of sent and received messages in the chat history. The content of each message, emojis, and external media files (images, videos, audio)
	should be logged without any loss or alteration. Edited or deleted messages should be reflected for at least 99.9\% of cases. \\
	{\bf Rationale:} Ensuring accurate log of chat history is crucial to maintaining data integrity. This ensures that users can trust the system to accurately
	store chat history, supporting efficient retrieval and review of past messages. A high accuracy is imperative for this application as files and messages should not be altered when stored in the database.
	\item The system shall remove the disappearing messages from the user interface within at least $\pm 1 second$ of the specified time limit. \\
	{\bf Rationale:} By defining precision for message removal, users are guaranteed that the messages will be removed from 
	the user interface at the specified time. A response time between 200 milliseconds and 1 second is considered acceptable and will not hinder user experience
	\cite{5c}. 
\end{enumerate}
% End SubSubSection

\subsubsection{Reliability and Availability Requirements}
\label{ssub:reliability_and_availability_requirements}
% Begin SubSubSection
\begin{enumerate}[{PR-RA}1. ]
	\item The system shall have an availability of 99.999\% ensuring that users can access the chat application without 
	significant downtime. \\
	{\bf Rationale:} A system that is continuously available is crucial to ensure real-time communication and that the service is accessible whenever
	the user needs it. As part of the concept of nines in service availability, a good value for availability is at least 99.999\%, ensuring 
	user contentment \cite{6c}.
\end{enumerate}
% End SubSubSection

\subsubsection{Robustness or Fault-Tolerance Requirements}
\label{ssub:robustness_or_fault_tolerance_requirements}
% Begin SubSubSection
\begin{enumerate}[{PR-RFT}1. ]
	\item The system shall remain functional when handling messages containing special characters, emojis, and
	multimedia content. The system will not crash when a diverse message format is encountered. \\
	{\bf Rationale:} Having a robust message handling prevents the application from becoming unresponsive and crashing 
	when faced with an unexpected message content.
	\item In the event of temporary network disconnection, the system shall have a fault-tolerant mechanism to store unsent messages locally and will automatically 
	attempt to resend the message once connection is reestablished. \\
	{\bf Rationale:} This ensures that no message is loss when the chat application is disconnected from the network, maintaining 
	a seamless user communication experience.
\end{enumerate}
% End SubSubSection

\subsubsection{Capacity Requirements}
\label{ssub:capacity_requirements}
% Begin SubSubSection
\begin{enumerate}[{PR-C}1. ]
	\item The system shall be able to support a minimum of 50 simultaneous users without degradation in performance and response time. \\
	{\bf Rationale:} Considering the scope of this project, the chat application should be able to accommodate a medium-size user base without 
	compromising the user experience. 
	\item The chat server shall provide a minimum of 1 terabyte storage capacity to securely store chat logs, media files, and other data. \\
	{\bf Rationale:} This requirement ensures that the application is capable of storing a significant amount of data without running out 
	of storage space. 
\end{enumerate} 
 
% End SubSubSection

\subsubsection{Scalability or Extensibility Requirements}
\label{ssub:scalability_or_extensibility_requirements}
% Begin SubSubSection
\begin{enumerate}[{PR-SE}1. ]
	\item The system must employ a modular architecture to enable seamless integration and extension of new features 
	without necessitating any major changes to the existing system. \\
	{\bf Rationale:} This requirement enables continuous development of the application while facilitating easy integration of new 
	features. This ensures that the application is capable of evolving with changing requirements 
	\cite{7c}.
\end{enumerate}
% End SubSubSection

\subsubsection{Longevity Requirements}
\label{ssub:longevity_requirements}
% Begin SubSubSection
\begin{enumerate}[{PR-L}1. ]
	\item The system shall remain compatible with Android operating system for a minimum of one year, ensuring continued 
	accessibility across evolving technology. \\
	{\bf Rationale:} This requirement ensures that the application can adapt to changes in technology and can maintain
	accessibility over an extended period of time (i.e., considering the scope of this course project).
	\item The system shall undergo regular software maintenance activities for a minimum of one year from initial release. \\
	{\bf Rationale:} Regular maintenance of the chat application ensures that the system remains secure and performs efficiently, and
	that new bugs and issues can be addressed, contributing to continued usefulness. The one year time frame considers the short duration 
	of this course project.
\end{enumerate}
% End SubSubSection

% End SubSection



\subsection{Operational and Environmental Requirements}
\label{sub:operational_and_environmental_requirements}
% Begin SubSection

\subsubsection{Expected Physical Environment}
\label{ssub:expected_physical_environment}
% Begin SubSubSection
\begin{enumerate}[{OE-EPE}1. ]
	\item N/A
\end{enumerate}
% End SubSubSection

\subsubsection{Requirements for Interfacing with Adjacent Systems}
\label{ssub:requirements_for_interfacing_with_adjacent_systems}
% Begin SubSubSection
\begin{enumerate}[{OE-IA}1. ]
	\item The application shall integrate with relevant APIs to fetch and display contextually relevant message response suggestions.\\
	{\bf Rationale:} Leveraging an external API enables the application to benefit from complex algorithms (Natural Language Processing and 
	machine learning models) allowing for more intelligent suggestions.
\end{enumerate} 

% End SubSubSection

\subsubsection{Productization Requirements}
\label{ssub:productization_requirements}
% Begin SubSubSection
\begin{enumerate}[{OE-P}1. ]
	\item N/A
\end{enumerate}
% End SubSubSection

\subsubsection{Release Requirements}
\label{ssub:release_requirements}
% Begin SubSubSection
\begin{enumerate}[{OE-R}1. ]
	\item The application shall be compatible with Android 6.0 or above. \\
	{\bf Rationale:} This ensures that the application can function on a wide variety of Android devices.
	As of 2021, over 70\% of Android users have a version of Android 6.0 and above, making it a widely
	used version of Android \cite{8c}.
\end{enumerate}
% End SubSubSection

% End SubSection



\subsection{Maintainability and Support Requirements}
\label{sub:maintainability_and_support_requirements}
% Begin SubSection

\subsubsection{Maintenance Requirements}
\label{ssub:maintenance_requirements}
% Begin SubSubSection
\begin{enumerate}[{MS-M}1. ]
	\item The development team shall address and fix any reported bugs or issues within the chat application.
	Any critical issues must be resolved within one week of report, while non-critical issues should be addressed within one month. \\
	{\bf Rationale:} Continuous resolution of issues is crucial in maintaining a stable and secure application,
	ensuring user satisfaction and confidence.
\end{enumerate}
% End SubSubSection

\subsubsection{Supportability Requirements}
\label{ssub:supportability_requirements}
% Begin SubSubSection
\begin{enumerate}[{MS-S}1. ]
	\item The system shall maintain a comprehensive frequently asked questions (FAQ) section to provide users with self-service
	resource for issue resolution. \\
	{\bf Rationale:} An FAQ would allow users to independently address common queries, reducing the dependency for direct 
	support.
\end{enumerate}
% End SubSubSection

\subsubsection{Adaptability Requirements}
\label{ssub:adaptability_requirements}
% Begin SubSubSection
\begin{enumerate}[{MS-A}1. ]
	\item The system shall be able to run on the most recent version of Android released on Android devices. \\
	{\bf Rationale:} The system should be able to run on Android devices as this is the platform that it is being designed for.
\end{enumerate}
% End SubSubSection

% End SubSection



\subsection{Security Requirements}
\label{sub:security_requirements}
% Begin SubSection

\subsubsection{Access Requirements}
\label{ssub:access_requirements}
% Begin SubSubSection
\begin{enumerate}[{SR-AC}1. ]
	\item The system shall enforce session timeouts to automatically log out users after a specified period of inactivity. \\
	{\bf Rationale:} Session timeout reduce the risk of unauthorized access in case the user forgets to log out.
	\item The system shall only allow users to access their accounts if the login credentials given by the user are correct. \\
	{\bf Rationale:} This requirement ensures that access to the user account is attained when correct login details are provided. 
	It serves as a means to authenticate users and adds a layer of security to user accounts, ensuring that only individuals with the 
	correct credentials can log in.
	\item Users shall only be able to access group chats in which they are listed as members. \\
	{\bf Rationale:} This requirement ensures that users cannot view or participate in any group chats that 
	they are not part of, ensuring privacy and security within the chat application.
	\item Regarding the disappearing message feature, users shall not be able to access the already read or seen message if it has exceeded the specified time limit
	after opening. \\
	{\bf Rationale:} Disappearing message help protect user privacy by removing sensitive or private information.
	Once the time limit has been reached, the user will not be able to access the information, reducing risk
	of unauthorized access.
\end{enumerate}
% End SubSubSection

\subsubsection{Integrity Requirements}
\label{ssub:integrity_requirements}
% Begin SubSubSection
\begin{enumerate}[{SR-INT}1. ]
	\item The application shall authenticate user to verify the identity of the user accessing the system. \\
	{\bf Rationale:} Authentication ensures that only authorized users can log in and use the application.
	\item The application shall ensure that even after removal of the disappearing message from the user interface,
	the removed message shall be securely stored in the database, maintaining their original content. \\
	{\bf Rationale:} This requirement ensures a tamper-resistant storage mechanism, preventing manipulation 
	of the already stored data.
\end{enumerate}
% End SubSubSection

\subsubsection{Privacy Requirements}
\label{ssub:privacy_requirements}
% Begin SubSubSection
\begin{enumerate}[{SR-P}1. ]
	\item The application shall obtain explicit consent from users before collecting, processing, or sharing sensitive personal information. \\
	{\bf Rationale:} Collection of explicit consent from the user will ensure that users are aware and agree to the specific use of their sensitive information. 
	This promotes transparency and helps users make informed decisions about their information. This requirement is part of Google Play Developer
	Distributor agreement \cite{9c}.
	\item The application shall have a mechanism for users to access and review their personal information stored by the application. \\
	{\bf Rationale:} Providing users access to their personal information foster transparency and allows users to verify the accuracy of their data.
	\item The system shall provide legally adequate privacy notice and protection for the users. This also involves the system using only
	user information for the sole purpose that the user has given.
	{\bf Rationale:} This requirement is part of Google Play Developer Distributor agreement \cite{9c}.
\end{enumerate}
% End SubSubSection

\subsubsection{Audit Requirements}
\label{ssub:audit_requirements}
% Begin SubSubSection
\begin{enumerate}[{SR-AU}1. ]
	\item The application shall store chat history log for each agent securely on the server. The log shall contain 
	identifiers of the communication agents, the date and time of the communication, and an accurate chat log. \\
	{\bf Rationale:} Storing chat history log allows for trail auditing for reviewing past conversations, tracking changes, 
	and investigating incidents.
\end{enumerate}
% End SubSubSection

\subsubsection{Immunity Requirements}
\label{ssub:immunity_requirements}
% Begin SubSubSection
\begin{enumerate}[{SR-IM}1. ]
	\item The system shall not accept unexpected input. \\
	{\bf Rationale:} This requirement aims to prevent attacks like SQL injections, where malicious input is used to manipulate or compromise the database. By rejecting such input, the system prevents unauthorized
	access attempts.
	\item  The system shall enforce strict content type validation for file uploads. \\
	{\bf Rationale:} Validation of content type of uploaded files can prevent attackers from exploiting vulnerabilities associated with improper handling of file uploads.
\end{enumerate}
% End SubSubSection

% End SubSection



\subsection{Cultural and Political Requirements}
\label{sub:cultural_and_political_requirements}
% Begin SubSection

\subsubsection{Cultural Requirements}
\label{ssub:cultural_requirements}
% Begin SubSubSection
\begin{enumerate}[{CP-C}1. ]
	\item The system shall integrate a calendar that highlights important holidays and events from different cultures. \\
	{\bf Rationale:} Storing calendar events from different cultures recognizes cultural diversity, helping users stay informed 
	of significant occasions. 
	\item The system prevent users from creating accounts with discriminatory names, based on unacceptable words. \\
	{\bf Rationale:} Preventing creation of discriminatory accounts will foster a professional environment within the workplace and 
	promotes responsible use.
\end{enumerate}
% End SubSubSection

\subsubsection{Political Requirements}
\label{ssub:political_requirements}
% Begin SubSubSection
\begin{enumerate}[{CP-P}1. ]
	\item N/A
\end{enumerate}
% End SubSubSection

% End SubSection





\subsection{Legal Requirements}
\label{sub:legal_requirements}
% Begin SubSection

\subsubsection{Compliance Requirements}
\label{ssub:compliance_requirements}
% Begin SubSubSection
\begin{enumerate}[{LR-COMP}1. ]
	\item The system shall comply with data protection laws through user consent for data processing, 
	providing privacy notices, and implementing measures to secure user data. \\
	{\bf Rationale:} User privacy is protected and ensures compliance to regulations governing the collection and processing
	of personal information.
	\item The system shall collect only the minimum amount of personal information necessary for stated purpose. \\
	{\bf Rationale:} This supports PIPEDA's principle of limiting the collection of personal information to what is 
	reasonable necessary \cite{10c}.
	\item The system shall only disclose or use user personal information for the purpose in which it was collected or 
	under the individual's consent.\\
	{\bf Rationale:} This supports PIPEDA's principle of limiting use, disclosure and retention, further protecting the 
	privacy of the user \cite{10c}.
	\item The system shall obtain the knowledge and consent of the user when collecting, using, 
	or disclosing personal information. \\
	{\bf Rationale:} This supports PIPEDA's principle of consent \cite{10c}.
	\item The system shall protect user personal information by appropriate security relative to the 
	sensitivity of the information. \\
	{\bf Rationale:} This supports PIPEDA's safeguard principle \cite{10c}.
\end{enumerate}

% End SubSubSection

\subsubsection{Standards Requirements}
\label{ssub:standards_requirements}
% Begin SubSubSection
\begin{enumerate}[{LR-STD}1. ]
	\item The application shall be designed to be user-friendly and accessible, aligning with legal standards of WCAG. \\
	{\bf Rationale:} Adherence to WCAG standards ensures legal compliance with inclusive design principle, making the application accessible to users with disabilities
	\cite{2c}.
	\item The application shall store all sensitive data in the application's internal storage. \\
	{\bf Rationale:} This is according to the User Data Policies of the Google Play Store
	\cite{11c}.
	\item The application shall use strong cryptographic algorithms and does not implement custom algorithms \\
	{\bf Rationale:} This is according to the User Data Policies of the Google Play Store
	\cite{11c}.
\end{enumerate}
% End SubSubSection

% End SubSection

% End Section

\appendix
\section{Division of Labour}
\label{sec:division_of_labour}
% Begin Section
Include a Division of Labour sheet which indicates the contributions of each team member. This sheet must be signed by all team members.
% End Section

%\newpage
%\section*{IMPORTANT NOTES}
%\begin{itemize}
%	\item Be sure to include all sections of the template in your document regardless whether you have something to write for each or not
%	\begin{itemize}
%		\item If you do not have anything to write in a section, indicate this by the \emph{N/A}, \emph{void}, \emph{none}, etc.
%	\end{itemize}
%	\item Uniquely number each of your requirements for easy identification and cross-referencing
%	\item Highlight terms that are defined in Section~1.3 (\textbf{Definitions, Acronyms, and Abbreviations}) with \textbf{bold}, \emph{italic} or \underline{underline}
%	\item For Deliverable 1, please highlight, in some fashion, all (you may have more than one) creative and innovative features. Your creative and innovative features will generally be described in Section~2.2 (\textbf{Product Functions}), but it will depend on the type of creative or innovative features you are including.
%\end{itemize}


\end{document}
%------------------------------------------------------------------------------
