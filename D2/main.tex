\documentclass[]{article}
% Imported Packages
%------------------------------------------------------------------------------
\usepackage{amssymb}
\usepackage{amstext}
\usepackage{amsthm}
\usepackage{amsmath}
\usepackage{enumerate}
\usepackage{float}
\usepackage{fancyhdr}
\usepackage[margin=1in]{geometry}
\usepackage{graphicx}
\graphicspath{{../signatures/}}
\usepackage{multirow}
%\usepackage{extarrows}
%\usepackage{setspace}
%------------------------------------------------------------------------------

% Header and Footer
%------------------------------------------------------------------------------
\pagestyle{plain}  
\renewcommand\headrulewidth{0.4pt}                                      
\renewcommand\footrulewidth{0.4pt}                                    
%------------------------------------------------------------------------------
% Title Details
%------------------------------------------------------------------------------
\title{Deliverable \#2 Template : Software Requirement Specification (SRS)}
\author{SE 3A04: Software Design II -- Large System Design}
\date{10 March 2024}
                            
%------------------------------------------------------------------------------

% Document
%------------------------------------------------------------------------------
\begin{document}

\maketitle
\noindent{\bf Tutorial Number:} T01\\
{\bf Group Number:} G07 \\
{\bf Group Members:}
\begin{itemize}
	\item Awurama Nyarko
	\item Chelsea Maramot
	\item Harrison Chiu
	\item Khushi Bhojane
	\item Sumanya Gulati
\end{itemize}

% \section*{IMPORTANT NOTES}
% \begin{itemize}
% 	%	\item You do \underline{NOT} need to provide a text explanation of each diagram; the diagram should speak for itself
% 	\item Please document any non-standard notations that you may have used
% 	\begin{itemize}
% 		\item \emph{Rule of Thumb}: if you feel there is any doubt surrounding the meaning of your notations, document them
% 	\end{itemize}
% 	\item Some diagrams may be difficult to fit into one page
% 	\begin{itemize}
% 		\item Ensure that the text is readable when printed, or when viewed at 100\% on a regular laptop-sized screen.
% 		\item If you need to break a diagram onto multiple pages, please adopt a system of doing so and thoroughly explain how it can be reconnected from one page to the next; if you are unsure about this, please ask about it
% 	\end{itemize}
% 	\item Please submit the latest version of Deliverable 1 with Deliverable 2
% 	\begin{itemize}
% 		\item Indicate any changes you made.
% 	\end{itemize}
% 	\item If you do \underline{NOT} have a Division of Labour sheet, your deliverable will \underline{NOT} be marked
% \end{itemize}

\newpage
\section{Introduction}
\label{sec:introduction}
% Begin Section

%This section should provide an brief overview of the entire document.

\subsection{Purpose}
\label{sub:purpose}
% Begin SubSection
The purpose of this document is to outline the architectural framework of the Secure Chat Android Application, highlighting the principal design elements and architectural choices. Intended for a broad spectrum of internal stakeholders, including project managers, developers, domain experts, and team members, this document aims to provide a comprehensive understanding of the system’s architecture. Familiarity with the Software Requirements Specification (SRS) and a foundational technical knowledge will enhance comprehension of the details presented herein.
% End SubSection

\subsection{System Description}
\label{sub:system_description}
% Begin SubSection
The Secure Chat Android Application is designed to facilitate encrypted communication among employees of an organization holding sensitive information and trade secrets. In light of increasing threats of corporate espionage, the application incorporates a Key Distribution Centre (KDC), mediated authentication protocols, and robust symmetric-key cryptography to ensure secure message exchange. Additionally, it maintains a secure log of chat histories, thereby meeting the organization's stringent security requirements. The system's architecture is crafted to support these functionalities efficiently, ensuring both security and usability.

% End SubSection

\subsection{Overview}
\label{sub:overview}
% Begin SubSection
Following this introduction, Section 2 unveils the Analysis Class Diagram, providing insights into the application's structure through a detailed class hierarchy. Section 3 delves into the architectural design, discussing the rationale behind the chosen architecture, the interplay between subsystems, and the design considerations that underpin the application's development. Lastly, Section 4 presents Class Responsibility Collaboration (CRC) Cards, offering a deeper exploration of classes, their responsibilities, and interactions, thereby rounding off the comprehensive architectural overview of the Secure Chat Android Application.

% End SubSection

% End Section

\section{Analysis Class Diagram}
\label{sec:analysis_class_diagram}
% Begin Section
%This section should provide an analysis class diagram for your application.
\begin{figure}[H]
	\centering
	\includegraphics[width=1\textwidth]{d2_analysis_class.jpg}
	\caption{Analysis Class Diagram}
\end{figure}
% End Section


\section{Architectural Design}
\label{sec:architectural_design}
% Begin Section
%This section should provide an overview of the overall architectural design of your application. Your overall architecture should show the division of the system into subsystems with high cohesion and low coupling.
\subsection{System Architecture}
\label{sub:system_architecture}
% Begin SubSection
\begin{itemize}
	\item SecureChat employs a layered architectural approach, specifically tailored for Android applications, to enhance modularity and scalability. This structured division into Presentation, Business Logic, and Data Access layers each address distinct functionalities within the app, facilitating a well-organized development environment and enabling specialized focus on each layer's responsibilities.
	
	\item The "Layered Architecture" pattern was selected due to its proven effectiveness in simplifying complex applications by segregating the system into manageable layers. This architecture not only allows for individual layer modifications without significant impact on others but also promotes reusability and separation of concerns, essential for maintaining a clean codebase and supporting agile development practices.
	
	\item Our choice of architecture is underpinned by several critical considerations. Foremost is security — a paramount concern for SecureChat. The layered approach enables us to implement robust security protocols at each level, particularly within the Business Logic and Data Access layers, where encryption and data integrity checks are crucial. Scalability, facilitated by this architecture, ensures the application can grow, both in terms of user base and functionality, without compromising performance. Adaptability, another key factor, allows for future technological advancements to be integrated seamlessly, ensuring the application remains cutting-edge.
	
	\item Layered architecture is great for a chat application because it organizes the system well, making it easier to work on and update. For example, there's a presentation layer for everything the user sees and interacts with, allowing changes to the look of the chat without messing with the main functions. The business logic layer deals with important tasks like managing chats through ChatManagement, keeping things secure with AuthenticationManager, and keeping messages private with EncryptionManager. These parts work together smoothly in their own layer. Then, there's a data access layer that handles all the data stuff, like saving and finding chat histories, making it easy to change how data is stored without affecting other parts of the app. This setup means you can fix or improve one part of the system without causing problems in others, making the whole app more reliable and easier to improve over time.

	\item Our exploration of architectural alternatives was exhaustive:
		\begin{itemize}
		    \item \textbf{Service-Oriented Architecture (SOA)} was initially attractive for its modular service components and potential for reusability across the application. However, our analysis revealed that the overhead associated with managing numerous service interfaces and the complexity of ensuring secure communication between them would outweigh the benefits for a project of SecureChat's scope and scale.
		    
		    \item \textbf{Microservices architecture} was considered for its scalability and the flexibility it offers for independent deployment of services. Despite these advantages, the distributed nature of microservices would introduce unnecessary complexity into the system, particularly regarding data consistency and transaction management, which are critical for SecureChat's reliability.
		    
		    \item \textbf{Event-Driven Architecture (EDA)} was explored for its responsiveness to real-time events, an appealing feature for a chat application. Nevertheless, the complexity of managing asynchronous communications and the potential difficulty in tracing and debugging event flows led us to conclude that EDA might complicate rather than streamline our development process.
		    
		    \item \textbf{Blackboard Architecture Style} was considered because of its similarity to the Repository style, with a focus on a large central data store. However, it was ultimately rejected because the data store would act as the active component rather than the clients. This setup would complicate the implementation for a chat application that needs active writing to the database and synchronization of multiple agents (e.g., chat users) to update shared data. Additionally, the blackboard architecture's ideal use case for solving specific problems did not align with the needs of our chat application.
		    
		    \item \textbf{Sequential Architecture Style} (batch sequential and pipe and filter) processes tasks in a linear, step-by-step manner and was found to be ill-suited for the overall architecture of a chat application. The need for real-time communication, simultaneous user interactions, and rapid response times in a chat application cannot be efficiently managed by a sequential approach. Moreover, essential features such as encryption, authentication, and message delivery require integration and decoupled operations for scalability, security, and performance. A sequential style would create bottlenecks, hinder performance, and limit the system's ability to scale, making it a poor choice for such a dynamic system.
		\end{itemize}
	    These architectures were meticulously evaluated against our stringent criteria of security, cohesion, ease of use, and maintainability. The insights gained from this comparative analysis reinforced our decision to adopt a unified, less fragmented architectural approach, ultimately guiding us to the Layered Architecture pattern. This decision aligns with our objectives to create a secure, efficient, and user-friendly chat application, capable of evolving in response to future demands without necessitating substantial architectural reconfiguration.
\end{itemize}

\begin{figure}[H]
	\centering
	\includegraphics[width=1\textwidth]{architecture_diagram.drawio.png}
	\caption{System Architecture of the Project}
\end{figure}
% End SubSection


\subsection{Subsystems}
\label{sub:subsystems}
% Begin SubSection
%  Provide a list of your subsystems, with a brief description of each. Be sure to document its purpose and relationship to other subsystems.

The Account Manager subsystem is central to user management operations within the application. It handles tasks such as user authentication, creating new user accounts, and maintaining user data. It interfaces with the Accounts Database (DB), which stores user information securely. By managing user identities (UIDs) and related data, it serves as the backbone for user verification and is critical for maintaining user integrity and security within the application. \\

The repository style is ideal for the Account Manager subsystem because it centralizes user data management, simplifying access and manipulation of account information. By acting as a bridge between the application logic and the data storage, it abstracts the complexities of database interactions, offering an easy interface for retrieving, updating, and storing user details. This approach enhances maintainability, as changes to the database schema or underlying storage mechanism can be isolated within the repository layer, minimizing impact on the rest of the application. Furthermore, it promotes data consistency and integrity, essential for managing sensitive account information, by encapsulating the logic for data access patterns and transactions. The repository style is good for the purpose of account management operations where multiple actors may try to update its data. \\

The Chat Manager is responsible for managing the chat sessions. It enables users to view all their chats, add or remove participants, and also manages the encryption keys from the Key Distribution Center (KDC). By interfacing with the Chat Messages DB, it ensures that all chat data is accurately reflected in user sessions. This subsystem works closely with the Message Manager, providing the necessary keys for encryption and decryption of messages. \\

Service-Oriented Architecture (SOA) style is best for ChatManager because it allows for flexible, scalable chat services. The ChatManager acts as a central hub for creating and managing chat sessions, easily integrating with other parts of the application and even external services. This subsystem encapsulates the logic for chat operations and exposes them as services to be used by clients. The modularity allows updates or new features can be added with minimal disruption. SOA's use of standardized protocols ensures seamless communication between different services, making it easier to maintain and extend the system. Additionally, the independence of services under SOA facilitates parallel development and deployment, speeding up the release cycle. \\

Message Manager: Serving as the communication facilitator, the message manager is tasked with handling all aspects of message exchange. It stores, encrypts, and decrypts messages, ensuring they are securely sent to and from users within the chat. This subsystem interacts with Chat Manager subsystem to get the necessary encryption and decryption keys. This is crucial for maintaining the confidentiality of the messages. The Message Manager works in conjunction with the Chat Manager to ensure messages within a chat are properly secured and delivered to the intended recipients. \\

The Message Manager subsystem aligns with the Pipe-and-Filter architectural style. This style has a chain of processors (filters) that does work on its input. Each element is the input to the next. In the case of Message Manager, it handles the flow of messages by performing operations such as validation, encryption/decryption, formatting, and ultimately delivering messages to their intended recipients. Each of these operations can be seen as a "filter" that processes the message data as it moves through the system's pipes. This architecture supports transformations on the message data, enforces message processing rules, and ensures that messages are correctly and securely managed and dispatched.\\

% End SubSection

% End Section
	
\section{Class Responsibility Collaboration (CRC) Cards}
\label{sec:class_responsibility_collaboration_crc_cards}
% Begin Section
%This section should contain all of your CRC cards.

%CRC 1 - Authentication Management (Controller)
	\begin{table}[ht]
		\centering
		\begin{tabular}{|p{7cm}|p{7cm}|}
		\hline 
		 \multicolumn{2}{|l|}{\textbf{Class Name:} Account Management (Controller)} \\
		\hline
		\textbf{Responsibility:} & \textbf{Collaborators:} \\
		\hline
		Knows Account Success & knows Account Success\\
		Knows Account Error & Account Error \\
		Knows Create Account & Create Account\\
		Knows Login & Login\\
		Knows Change Status & Change Status\\
		Knows View Account & View Account\\
		Knows Account Database & Account Database\\
		Knows User Information &User Information\\
		Handles overall account-related functionalities
		\vspace{0.1in} & \\
		\hline
		\end{tabular}
	\end{table}

	\begin{table}[ht]
		\centering
		\begin{tabular}{|p{7cm}|p{7cm}|}
		\hline 
		 \multicolumn{2}{|l|}{\textbf{Class Name:} Account Success (Boundary)} \\
		\hline
		\textbf{Responsibility:} & \textbf{Collaborators:} \\
		\hline
		Knows Account Management & Account Management\\
		Handles unsuccessful events of saving and creating account
		\vspace{0.1in} & \\
		\hline
		\end{tabular}
	\end{table}

	\begin{table}[ht]
		\centering
		\begin{tabular}{|p{7cm}|p{7cm}|}
		\hline 
		 \multicolumn{2}{|l|}{\textbf{Class Name:} Account Error (Boundary)} \\
		\hline
		\textbf{Responsibility:} & \textbf{Collaborators:} \\
		\hline
		Knows Account Management & Account Management \\
		Knows Error Type &\\
		Knows Error Details &\\
		Handles account time-out event &\\
		Handles Creating Account error event &\\
		\vspace{0.1in} & \\
		\hline
		\end{tabular}
	\end{table}

	\begin{table}[ht]
		\centering
		\begin{tabular}{|p{7cm}|p{7cm}|}
		\hline 
		 \multicolumn{2}{|l|}{\textbf{Class Name:} Create Account (Boundary)} \\
		\hline
		\textbf{Responsibility:} & \textbf{Collaborators:} \\
		\hline
			Knows Account Management  & Account Management\\
			Knows Username &\\
			Knows Password &\\
			Knows Authorization and Permission Details &\\
			Handles validation of user-provided information for creating an account &\\
			Handles click-event of “Create Account” Button &\\
		\vspace{0.1in} & \\
		\hline
		\end{tabular}
	\end{table}

	\begin{table}[ht]
		\centering
		\begin{tabular}{|p{7cm}|p{7cm}|}
		\hline 
		 \multicolumn{2}{|l|}{\textbf{Class Name:} Edit Account (Boundary)} \\
		\hline
		\textbf{Responsibility:} & \textbf{Collaborators:} \\
		\hline
			Knows Account Management & Account Management\\
			Knows Username &\\
			Knows Password &\\
			Knows Authorization and Permission Details &\\
			Handles click-event of “Save” Button &\\
		\vspace{0.1in} & \\
		\hline
		\end{tabular}
	\end{table}

	\begin{table}[ht]
		\centering
		\begin{tabular}{|p{7cm}|p{7cm}|}
		\hline 
		 \multicolumn{2}{|l|}{\textbf{Class Name:} View Account(Boundary)} \\
		\hline
		\textbf{Responsibility:} & \textbf{Collaborators:} \\
		\hline
			Knows Account Management & Account Management \\
			Handles display of user account details to the user interface &\\
			Handles click-event of “View Account” button &\\
		\vspace{0.1in} & \\
		\hline
		\end{tabular}
	\end{table}

	\begin{table}[ht]
		\centering
		\begin{tabular}{|p{7cm}|p{7cm}|}
		\hline 
		 \multicolumn{2}{|l|}{\textbf{Class Name:} Login(Boundary)} \\
		\hline
		\textbf{Responsibility:} & \textbf{Collaborators:} \\
		\hline
			Knows Account Management & Account Management\\
			Knows Authentication and Authorization details & Authentication Management\\
			Handles validation of user login credentials &\\
			Handles click event of “login” button &\\
		\vspace{0.1in} & \\
		\hline
		\end{tabular}
	\end{table}

	\begin{table}[ht]
		\centering
		\begin{tabular}{|p{7cm}|p{7cm}|}
		\hline 
		 \multicolumn{2}{|l|}{\textbf{Class Name:} Current Status (Boundary)} \\
		\hline
		\textbf{Responsibility:} & \textbf{Collaborators:} \\
		\hline
			Knows Account Management & Account Management \\
			Knows User Availability Status (e.g., online, away, do not disturb, custom, in a meeting) &\\
			Handles updates of current User Availability Status &\\
		\vspace{0.1in} & \\
		\hline
		\end{tabular}
	\end{table}

	\begin{table}[ht]
		\centering
		\begin{tabular}{|p{7cm}|p{7cm}|}
		\hline 
		 \multicolumn{2}{|l|}{\textbf{Class Name:} Account Database (Entity)} \\
		\hline
		\textbf{Responsibility:} & \textbf{Collaborators:} \\
		\hline
			Knows Account Management & Account Management \\
			Knows User Information & User Information \\
			Knows Account Permissions &\\
			Knows Account Status &\\
			Handles storage of user account data &\\
		\vspace{0.1in} & \\
		\hline
		\end{tabular}
	\end{table}

	\begin{table}[ht]
		\centering
		\begin{tabular}{|p{7cm}|p{7cm}|}
		\hline 
		 \multicolumn{2}{|l|}{\textbf{Class Name:} User Information (Entity)} \\
		\hline
		\textbf{Responsibility:} & \textbf{Collaborators:} \\
		\hline

			Knows Account Management & Account Management \\
			Knows Account Database & Account Database \\
			Knows Username &\\
			Knows Password &\\
			Knows Gender &\\
			Knows Team Manager &\\
			Knows Company Name &\\
			Knows Date-of-Birth &\\
			Handles storage and management of user specific details &\\
		\vspace{0.1in} & \\
		\hline
		\end{tabular}
	\end{table}

	\begin{table}[ht]
		\centering
		\begin{tabular}{|p{7cm}|p{7cm}|}
		\hline 
		 \multicolumn{2}{|l|}{\textbf{Class Name:} File Management (Controller)} \\
		\hline
		\textbf{Responsibility:} & \textbf{Collaborators:} \\
		\hline
			Knows File Name & Chat Management Controller\\
			Knows File Size & File Database\\
			Knows File Type &\\
			Knows File Permissions &\\
			Handles coordination of sending, receiving, and storage of files &\\
			Handles the enforcement of file access permission &\\
		\vspace{0.1in} & \\
		\hline
		\end{tabular}
	\end{table}


	\begin{table}[ht]
		\centering
		\begin{tabular}{|p{7cm}|p{7cm}|}
		\hline 
		 \multicolumn{2}{|l|}{\textbf{Class Name:} File Database (Entity)} \\
		\hline
		\textbf{Responsibility:} & \textbf{Collaborators:} \\
		\hline
			Knows File Management & File Management \\
			Knows File Name &\\
			Knows File Size &\\
			Knows File Type &\\
			Knows File Permissions &\\
			Knows File Owner &\\
			Handles the storage of file data &\\
		\vspace{0.1in} & \\
		\hline
		\end{tabular}
	\end{table}


	\begin{table}[ht]
		\centering
		\begin{tabular}{|p{7cm}|p{7cm}|}
		\hline 
		 \multicolumn{2}{|l|}{\textbf{Class Name:} File Search (Boundary)} \\
		\hline
		\textbf{Responsibility:} & \textbf{Collaborators:} \\
		\hline
			Knows File Management & File Management\\
			Knows File Search Criteria (e.g., File Name, File Type, File Owner) &\\
			Handles file search functionality &\\
			Handles utilization of search criteria to locate and retrieve files matching specified conditions &\\
			Handles click-event of “Search” button &\\
		\vspace{0.1in} & \\
		\hline
		\end{tabular}
	\end{table}

	\begin{table}[ht]
		\centering
		\begin{tabular}{|p{7cm}|p{7cm}|}
		\hline 
		 \multicolumn{2}{|l|}{\textbf{Class Name:} File Retrieval (Boundary)} \\
		\hline
		\textbf{Responsibility:} & \textbf{Collaborators:} \\
		\hline
			Knows File Management & File Management \\
			Knows File ID & File Search \\
			Knows File Name &\\
			Knows File Type &\\
			Knows File Owner &\\
			Handles retrieval of files from storage based on specified file search criteria &\\
		\vspace{0.1in} & \\
		\hline
		\end{tabular}
	\end{table}

	\begin{table}[ht]
		\centering
		\begin{tabular}{|p{7cm}|p{7cm}|}
		\hline 
		 \multicolumn{2}{|l|}{\textbf{Class Name:} File Error (Boundary)} \\
		\hline
		\textbf{Responsibility:} & \textbf{Collaborators:} \\
		\hline
			Knows File Management & File Management \\
			Knows Error Type &\\
			Knows Error Details &\\
			Handles errors related to file operations (e.g., Invalid File Type Error, File Size Too Big) &\\
			Handles communication of error information to the user &\\
		\vspace{0.1in} & \\
		\hline
		\end{tabular}
	\end{table}

	\begin{table}[ht]
		\centering
		\begin{tabular}{|p{7cm}|p{7cm}|}
		\hline 
		\multicolumn{2}{|l|}{\textbf{Class Name:} Chat History Database (Entity)} \\
		\hline
		\textbf{Responsibility:} & \textbf{Collaborators:} \\
		\hline
  			Knows Chat Management Controller & Chat Management Controller \\
     			Knows User Info & User Info \\
			Knows Send Message & Send Message \\
			Knows Send Files & Send Files \\
			Knows Delete/Edit Message & Delete/Edit Message\\
			Knows Disappearing Message & Disappearing Message \\
			Handles Load Previous Message events & \\
		\vspace{1in} & \\
		\hline
		\end{tabular}
	\end{table}

  	\begin{table}[ht]
		\centering
		\begin{tabular}{|p{7cm}|p{7cm}|}
		\hline 
		\multicolumn{2}{|l|}{\textbf{Class Name:} Create Group (Boundary)} \\
		\hline
		\textbf{Responsibility:} & \textbf{Collaborators:} \\
		\hline
  			Knows Chat Management Controller & Chat Management Controller \\
			Knows Edit Group Members & Edit Group Members \\
			Knows Edit Group Details & Edit Group Details \\
			Knows User Info & User Info \\
			Handles new group creation events &\\
		\vspace{1in} & \\
		\hline
		\end{tabular}
	\end{table}

   	\begin{table}[ht]
		\centering
		\begin{tabular}{|p{7cm}|p{7cm}|}
		\hline 
		\multicolumn{2}{|l|}{\textbf{Class Name:} Edit Group Members (Boundary)} \\
		\hline
		\textbf{Responsibility:} & \textbf{Collaborators:} \\
		\hline
  			Knows Chat Management Controller & Chat Management Controller \\
			Knows User Info & User Info \\
			Handles events pertaining to the addition/removal of members from a group chat &\\
		\vspace{1in} & \\
		\hline
		\end{tabular}
	\end{table}

   	\begin{table}[ht]
		\centering
		\begin{tabular}{|p{7cm}|p{7cm}|}
		\hline 
		\multicolumn{2}{|l|}{\textbf{Class Name:} Edit Group Details (Boundary)} \\
		\hline
		\textbf{Responsibility:} & \textbf{Collaborators:} \\
		\hline
  			Knows Chat Management Controller & Chat Management Controller \\
			Knows Create Group & Create Group \\
			Handles events related to editing group chat details including name, description and picture. &\\
		\vspace{1in} & \\
		\hline
		\end{tabular}
	\end{table}

    	\begin{table}[ht]
		\centering
		\begin{tabular}{|p{7cm}|p{7cm}|}
		\hline 
		\multicolumn{2}{|l|}{\textbf{Class Name:} Message Error (Boundary)} \\
		\hline
		\textbf{Responsibility:} & \textbf{Collaborators:} \\
		\hline
  			Knows Chat Management Controller & Chat Management Controller \\
			Knows Send Message & Send Message \\
			Knows Delete/Edit Message & Delete/Edit Message \\
			Knows Disappearing Message & Disappearing Message \\
			Knows Delivered Status Message & Delivered Status Message \\
   			Handles unsuccessful events of sending a message. &\\
			Handles unsuccessful events of deleting or editing a message past the set time limit. &\\
			Handles unsuccessful events of viewing a disappearing message after it has been read. &\\
		\vspace{1in} & \\
		\hline
		\end{tabular}
	\end{table}

    	\begin{table}[ht]
		\centering
		\begin{tabular}{|p{7cm}|p{7cm}|}
		\hline 
		\multicolumn{2}{|l|}{\textbf{Class Name:} Chat Management Controller} \\
		\hline
		\textbf{Responsibility:} & \textbf{Collaborators:} \\
		\hline
  			Knows File Management Controller & File Management Controller \\
			Knows Send Message & Send Message \\
			Knows Send Files & Send Files \\
			Knows Delete/Edit Message & Delete/Edit Message \\
			Knows Disappearing Message & Disappearing Message \\
			Knows Load Previous Message & Load Previous Message \\
			Knows Delivered Status Message & Delivered Status Message \\
			Knows Chat History Database & Chat History Database \\
			Knows Create Group & Create Group \\
			Knows Edit Group Members & Edit Group Members \\
			Knows Edit Group Details & Edit Group Details \\
			Knows Message Error & Message Error \\
		\vspace{1in} & \\
		\hline
		\end{tabular}
	\end{table}

    	\begin{table}[ht]
		\centering
		\begin{tabular}{|p{7cm}|p{7cm}|}
		\hline 
		\multicolumn{2}{|l|}{\textbf{Class Name:} File Permissions (Boundary)} \\
		\hline
		\textbf{Responsibility:} & \textbf{Collaborators:} \\
		\hline
  			Knows File Management Controller & File Management Controller \\
			Knows File Name & File Name \\
			Knows File Owner & File Owner \\
			Knows File Database & File Database \\
			Knows File Access & File Access \\
			Knows File Error Handling & File Error Handling \\
			Handles setting permission for file access events. &\\
		\vspace{1in} & \\
		\hline
		\end{tabular}
	\end{table}

     	\begin{table}[ht]
		\centering
		\begin{tabular}{|p{7cm}|p{7cm}|}
		\hline 
		\multicolumn{2}{|l|}{\textbf{Class Name:} Authentication Management (Controller)} \\
		\hline
		\textbf{Responsibility:} & \textbf{Collaborators:} \\
		\hline
            Knows Account Management & Account Management\\
            
            Knows User Information & User Information\\
            Knows Username & Encryption\\
            Knows Password\
            
            \vspace{0.1in}
            \textbf{Handles token generation which will be used by the user to access their account.}

		\vspace{1in} & \\
		\hline
  
		\end{tabular}
	\end{table}
 

 %CRC 2 - Token Generation (Entity)
	\begin{table}[ht]
		\centering
		\begin{tabular}{|p{7cm}|p{7cm}|}
		\hline 
		 \multicolumn{2}{|l|}{\textbf{Class Name:} Token Generation (Entity)} \\
		\hline
		\textbf{Responsibility:} & \textbf{Collaborators:} \\
		\hline
            Knows Account Management & Authentication Management\\
            
            Knows User Information & Account Management\\
            & Encryption\\
            
            \vspace{0.1in}
            \textbf{Handles the generation of a token which will be used by the user to access their account.}

		\vspace{1in} & \\
		\hline
  
		\end{tabular}
	\end{table}

 %CRC 3 - Encryption (Boundary)
	\begin{table}[ht]
		\centering
		\begin{tabular}{|p{7cm}|p{7cm}|}
		\hline 
		 \multicolumn{2}{|l|}{\textbf{Class Name:}  Encryption (Boundary)} \\
		\hline
		\textbf{Responsibility:} & \textbf{Collaborators:} \\
		\hline
            Knows Chat Management & Authentication Management\\
            Knows User Information & Account Management\\
            Knows Authentication Management & User Information\\
            Knows Account Management
            
            \vspace{0.1in}
            \textbf{Handles the overall security and encryption of user data and messages being sent as well as company data.}

		\vspace{1in} & \\
		\hline
  
		\end{tabular}
	\end{table}
	\begin{table}[ht]
		\centering
		\begin{tabular}{|p{7cm}|p{7cm}|}
		\hline 
		 \multicolumn{2}{|l|}{\textbf{Class Name:} Send Message (Boundary)} \\
		\hline
		\textbf{Responsibility:} & \textbf{Collaborators:} \\
		\hline
            Knows Message Contents and Type & Chat Management\\
            Knows Message Permissions & File Management\\
            & Encryption\\
            
            \vspace{0.1in}
            \textbf{Handles the sending of messages between accounts.}

		\vspace{1in} & \\
		\hline
  
		\end{tabular}
	\end{table}

	\begin{table}[ht]
		\centering
		\begin{tabular}{|p{7cm}|p{7cm}|}
		\hline 
		 \multicolumn{2}{|l|}{\textbf{Class Name:} Send Files (Boundary)} \\
		\hline
		\textbf{Responsibility:} & \textbf{Collaborators:} \\
		\hline
            Knows Name & File Management\\
            Knows File Size & Chat Management\\
            Knows File Type & Encryption\\
            Knows File Permissions\\
            
            \vspace{0.1in}
            \textbf{Handles the sending of files between accounts.}

		\vspace{1in} & \\
		\hline
  
		\end{tabular}
	\end{table}

	\begin{table}[ht]
		\centering
		\begin{tabular}{|p{7cm}|p{7cm}|}
		\hline 
		 \multicolumn{2}{|l|}{\textbf{Class Name:} Delete/Edit Messages (Boundary)} \\
		\hline
		\textbf{Responsibility:} & \textbf{Collaborators:} \\
		\hline
            Knows Message Contents and Type & Chat Management\\
            Knows Message Permissions & File Management\\
            & Encryption\\
            
            \vspace{0.1in}
            \textbf{Handles the deletion and editing of messages between accounts.}

		\vspace{1in} & \\
		\hline
  
		\end{tabular}
	\end{table}
	\begin{table}[ht]
		\centering
		\begin{tabular}{|p{7cm}|p{7cm}|}
		\hline 
		 \multicolumn{2}{|l|}{\textbf{Class Name:} Disappearing Messages - Set Time (Boundary)} \\
		\hline
		\textbf{Responsibility:} & \textbf{Collaborators:} \\
		\hline
            Knows Message Contents and Type & Chat Management\\
            Knows Message Permissions & File Management\\
            & Encryption\\
            
            \vspace{0.1in}
            \textbf{Handles the disappearing of messages after a set time.}

		\vspace{1in} & \\
		\hline
  
		\end{tabular}
	\end{table}
	\begin{table}[ht]
		\centering
		\begin{tabular}{|p{7cm}|p{7cm}|}
		\hline 
		 \multicolumn{2}{|l|}{\textbf{Class Name:} Load Previous Message (Boundary)} \\
		\hline
		\textbf{Responsibility:} & \textbf{Collaborators:} \\
		\hline
            Knows Message Contents and Type & Chat Management\\
            Knows Message Permissions & File Management\\
            & Encryption\\
            
            \vspace{0.1in}
            \textbf{Handles the storage and loading of previous messages.}

		\vspace{1in} & \\
		\hline
  
		\end{tabular}
	\end{table}
	\begin{table}[ht]
		\centering
		\begin{tabular}{|p{7cm}|p{7cm}|}
		\hline 
		 \multicolumn{2}{|l|}{\textbf{Class Name:} Delivered Status Message (Boundary)} \\
		\hline
		\textbf{Responsibility:} & \textbf{Collaborators:} \\
		\hline
            Knows Message Contents and Type & Chat Management\\
            Knows Message Permissions & Encryption\\
            \vspace{0.1in}
            \textbf{Handles the storage and loading of previous messages.}

		\vspace{1in} & \\
		\hline
		\end{tabular}
	\end{table}



% End Section
\clearpage

\clearpage

\appendix
\section{Division of Labour}
\label{sec:division_of_labour}
% Begin Section
Include a Division of Labour sheet which indicates the contributions of each team member. This sheet must be signed by all team members.
% End Section
\subsection{Awurama Nyarko}
\label{subsec:awurama_nyarko}
\begin{itemize}
	\item 1.1 Purpose
	\item 1.2 System Description
	\item 1.3 Overview
	\item 3.1 System Architecture: explained used of MVC and Repository
\end{itemize}
\includegraphics[width=0.5\textwidth]{awurama.jpg}

\subsection{Chelsea Maramot}
\label{subsec:chelsea_maramot}
\begin{itemize}
	\item Figure 1. Part of Analysis Class Diagram
	\item Class Responsibility Collaboration (CRC) cards
 		\begin{itemize}
   			\item File Management: File Error Handling
      			\item File Management: File Access
	 		\item File Management: File Search and Retrieval
    			\item Account Management: Account Management
       			\item Account Management: Account Database
	  		\item Account Management: User Information
     			\item Account Management: Log-in/Sign-in
			\item Account Management: Change Status
   			\item Account Management: View Account
      		\end{itemize}
\end{itemize}
\includegraphics[width=0.5\textwidth]{chelsea.png}

\subsection{Harrison Chiu}
\label{subsec:harrison_chiu}
\begin{itemize}
	\item 3.1 System Architetcure: explained elimiation of other architecture designs
	\item Figure 2. System Architecture
 	\item 3.2 Subsystems
\end{itemize}
\includegraphics[width=0.5\textwidth]{harrison.png}

\subsection{Khushi Bhojane}
\label{subsec:khushi_bhojane}
\begin{itemize}
	\item Figure 1. Part of Analysis Class Diagram
	\item Class Responsibility Collaboration (CRC) cards
 		\begin{itemize}
   			\item Authentication Management: Encryption
      			\item Authentication Management: Token Generation
	 		\item Authentication Management: Authentication Management
    			\item Chat Management: Send Message
       			\item Chat Management: Send Files
	  		\item Chat Management: Edit/Delete Message
     			\item Chat Management: Disappearing/Vanishing Messages
			\item Chat Management: Load Previous Messages
   			\item Chat Management: 'Delivered' Status Message
      		\end{itemize}
\end{itemize}
\includegraphics[width=0.5\textwidth]{khushi_signature.png}

\subsection{Sumanya Gulati}
\label{subsec:sumanya_gulati}
\begin{itemize}
	\item Figure 1. Part of Analysis Class Diagram
	\item Class Responsibility Collaboration (CRC) cards
 		\begin{itemize}
   			\item Chat Management: Chat History Database
      			\item Chat Management: Create Group
	 		\item Chat Management: Edit Group Members
    			\item Chat Management: Edit Group Details
       			\item Chat Management: Message Error
	  		\item Chat Management: Chat Management
     			\item File Management: File Management
			\item File Management: File Permissions
      		\end{itemize}
\end{itemize}
\includegraphics[width=0.5\textwidth]{signature.jpeg}

\end{document}
%------------------------------------------------------------------------------
