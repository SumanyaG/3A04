\documentclass[]{article}

% Imported Packages
%------------------------------------------------------------------------------
\usepackage{amssymb}
\usepackage{amstext}
\usepackage{amsthm}
\usepackage{amsmath}
\usepackage{enumerate}
\usepackage{fancyhdr}
\usepackage[margin=1in]{geometry}
\usepackage{graphicx}
\usepackage{multirow}
%\usepackage{extarrows}
%\usepackage{setspace}
%------------------------------------------------------------------------------

% Header and Footer
%------------------------------------------------------------------------------
\pagestyle{plain}  
\renewcommand\headrulewidth{0.4pt}                                      
\renewcommand\footrulewidth{0.4pt}                                    
%------------------------------------------------------------------------------

% Title Details
%------------------------------------------------------------------------------
\title{Deliverable \#2 Template}
\author{SE 3A04: Software Design II -- Large System Design}
\date{}                               
%------------------------------------------------------------------------------

% Document
%------------------------------------------------------------------------------
\begin{document}

\maketitle	
\noindent{\bf Tutorial Number:} T0x\\
{\bf Group Number:} Gx \\
{\bf Group Members:} 
\begin{itemize}
	\item List all Group Member Names (as listed in Avenue)
	\item You do not need to use student \#s or macid (keep those private).
\end{itemize}

\section*{IMPORTANT NOTES}
\begin{itemize}
	%	\item You do \underline{NOT} need to provide a text explanation of each diagram; the diagram should speak for itself
	\item Please document any non-standard notations that you may have used
	\begin{itemize}
		\item \emph{Rule of Thumb}: if you feel there is any doubt surrounding the meaning of your notations, document them
	\end{itemize}
	\item Some diagrams may be difficult to fit into one page
	\begin{itemize}
		\item Ensure that the text is readable when printed, or when viewed at 100\% on a regular laptop-sized screen.
		\item If you need to break a diagram onto multiple pages, please adopt a system of doing so and thoroughly explain how it can be reconnected from one page to the next; if you are unsure about this, please ask about it
	\end{itemize}
	\item Please submit the latest version of Deliverable 1 with Deliverable 2
	\begin{itemize}
		\item Indicate any changes you made.
	\end{itemize}
	\item If you do \underline{NOT} have a Division of Labour sheet, your deliverable will \underline{NOT} be marked
\end{itemize}

\newpage
\section{Introduction}
\label{sec:introduction}
% Begin Section

This section should provide an brief overview of the entire document.

\subsection{Purpose}
\label{sub:purpose}
% Begin SubSection
State the purpose and intended audience for the document.
% End SubSection

\subsection{System Description}
\label{sub:system_description}
% Begin SubSection
Give a brief description of the system. This could be a paragraph or two to give some context to this document.

% End SubSection

\subsection{Overview}
\label{sub:overview}
% Begin SubSection
Describe what the rest of the document contains and explain how the document is organised (e.g. "In Section 2 we discuss...in Section 3...").

% End SubSection

% End Section

\section{Analysis Class Diagram}
\label{sec:analysis_class_diagram}
% Begin Section
This section should provide an analysis class diagram for your application.
% End Section


\section{Architectural Design}
\label{sec:architectural_design}
% Begin Section
This section should provide an overview of the overall architectural design of your application. Your overall architecture should show the division of the system into subsystems with high cohesion and low coupling.

\subsection{System Architecture}
\label{sub:system_architecture}
% Begin SubSection
\begin{itemize}
	\item Identify and explain the overall architecture of your system
	\item Be sure to clearly state the name of the architecture you used (this is the name of the architectural pattern, not the name of your system)
	\item Provide the reasoning and justification of the choice of architecture
	\item Provide a structural architecture diagram showing the relationship among the subsystems (if appropriate)
	\item List any design alternatives you considered, but eliminated (and explain why you eliminated them)
\end{itemize}
% End SubSection

\subsection{Subsystems}
\label{sub:subsystems}
% Begin SubSection
 Provide a list of your subsystems, with a brief description of each. Be sure to document its purpose and relationship to other subsystems.

% End SubSection

% End Section
	
\section{Class Responsibility Collaboration (CRC) Cards}
\label{sec:class_responsibility_collaboration_crc_cards}
% Begin Section
This section should contain all of your CRC cards.

\begin{itemize}
	\item Provide a CRC Card for each identified class
	\item Please use the format outlined in tutorial, i.e., 
	\begin{table}[ht]
		\centering
		\begin{tabular}{|p{5cm}|p{5cm}|}
		\hline 
		 \multicolumn{2}{|l|}{\textbf{Class Name:} Account Management (Controller)} \\
		\hline
		\textbf{Responsibility:} & \textbf{Collaborators:} \\
		\hline
		Knows Account Success & knows Account Success\\
		Knows Account Error & Account Error \\
		Knows Create Account & Create Account\\
		Knows Login & Login\\
		Knows Change Status & Change Status\\
		Knows View Account & View Account\\
		Knows Account Database & Account Database\\
		Knows User Information &User Information\\
		Handles overall account-related functionalities
		\vspace{0.1in} & \\
		\hline
		\end{tabular}
	\end{table}

	\begin{table}[ht]
		\centering
		\begin{tabular}{|p{5cm}|p{5cm}|}
		\hline 
		 \multicolumn{2}{|l|}{\textbf{Class Name:} Account Success (Boundary)} \\
		\hline
		\textbf{Responsibility:} & \textbf{Collaborators:} \\
		\hline
		Knows Account Management & Account Management\\
		Handles unsuccessful events of saving and creating account
		\vspace{0.1in} & \\
		\hline
		\end{tabular}
	\end{table}

	\begin{table}[ht]
		\centering
		\begin{tabular}{|p{5cm}|p{5cm}|}
		\hline 
		 \multicolumn{2}{|l|}{\textbf{Class Name:} Account Error (Boundary)} \\
		\hline
		\textbf{Responsibility:} & \textbf{Collaborators:} \\
		\hline
		Knows Account Management & Account Management \\
		Knows Error Type &\\
		Knows Error Details &\\
		Handles account time-out event &\\
		Handles Creating Account error event &\\
		\vspace{0.1in} & \\
		\hline
		\end{tabular}
	\end{table}

	\begin{table}[ht]
		\centering
		\begin{tabular}{|p{5cm}|p{5cm}|}
		\hline 
		 \multicolumn{2}{|l|}{\textbf{Class Name:} Create Account (Boundary)} \\
		\hline
		\textbf{Responsibility:} & \textbf{Collaborators:} \\
		\hline
			Knows Account Management  & Account Management\\
			Knows Username &\\
			Knows Password &\\
			Knows Authorization and Permission Details &\\
			Handles validation of user-provided information for creating an account &\\
			Handles click-event of “Create Account” Button &\\
		\vspace{0.1in} & \\
		\hline
		\end{tabular}
	\end{table}

	\begin{table}[ht]
		\centering
		\begin{tabular}{|p{5cm}|p{5cm}|}
		\hline 
		 \multicolumn{2}{|l|}{\textbf{Class Name:} Edit Account (Boundary)} \\
		\hline
		\textbf{Responsibility:} & \textbf{Collaborators:} \\
		\hline
			Knows Account Management & Account Management\\
			Knows Username &\\
			Knows Password &\\
			Knows Authorization and Permission Details &\\
			Handles click-event of “Save” Button &\\
		\vspace{0.1in} & \\
		\hline
		\end{tabular}
	\end{table}

	\begin{table}[ht]
		\centering
		\begin{tabular}{|p{5cm}|p{5cm}|}
		\hline 
		 \multicolumn{2}{|l|}{\textbf{Class Name:} View Account(Boundary)} \\
		\hline
		\textbf{Responsibility:} & \textbf{Collaborators:} \\
		\hline
			Knows Account Management & Account Management \\
			Handles display of user account details to the user interface &\\
			Handles click-event of “View Account” button &\\
		\vspace{0.1in} & \\
		\hline
		\end{tabular}
	\end{table}

	\begin{table}[ht]
		\centering
		\begin{tabular}{|p{5cm}|p{5cm}|}
		\hline 
		 \multicolumn{2}{|l|}{\textbf{Class Name:} Login(Boundary)} \\
		\hline
		\textbf{Responsibility:} & \textbf{Collaborators:} \\
		\hline
			Knows Account Management & Account Management\\
			Knows Authentication and Authorization details &\\
			Handles validation of user login credentials &\\
			Handles click event of “login” button &\\
		\vspace{0.1in} & \\
		\hline
		\end{tabular}
	\end{table}

	\begin{table}[ht]
		\centering
		\begin{tabular}{|p{5cm}|p{5cm}|}
		\hline 
		 \multicolumn{2}{|l|}{\textbf{Class Name:} Current Status (Boundary)} \\
		\hline
		\textbf{Responsibility:} & \textbf{Collaborators:} \\
		\hline
			Knows Account Management & Account Management \\
			Knows User Availability Status (e.g., online, away, do not disturb, custom, in a meeting) &\\
			Handles updates of current User Availability Status &\\
		\vspace{0.1in} & \\
		\hline
		\end{tabular}
	\end{table}

	\begin{table}[ht]
		\centering
		\begin{tabular}{|p{5cm}|p{5cm}|}
		\hline 
		 \multicolumn{2}{|l|}{\textbf{Class Name:} Account Database (Entity)} \\
		\hline
		\textbf{Responsibility:} & \textbf{Collaborators:} \\
		\hline
			Knows Account Management & Account Management \\
			Knows User Information & User Information \\
			Knows Account Permissions &\\
			Knows Account Status &\\
			Handles storage of user account data &\\
		\vspace{0.1in} & \\
		\hline
		\end{tabular}
	\end{table}

	\begin{table}[ht]
		\centering
		\begin{tabular}{|p{5cm}|p{5cm}|}
		\hline 
		 \multicolumn{2}{|l|}{\textbf{Class Name:} User Information (Entity)} \\
		\hline
		\textbf{Responsibility:} & \textbf{Collaborators:} \\
		\hline

			Knows Account Management & Account Management \\
			Knows Account Database & Account Database \\
			Knows Username &\\
			Knows Password &\\
			Knows Gender &\\
			Knows Team Manager &\\
			Knows Company Name &\\
			Knows Date-of-Birth &\\
			Handles storage and management of user specific details &\\
		\vspace{0.1in} & \\
		\hline
		\end{tabular}
	\end{table}

	\begin{table}[ht]
		\centering
		\begin{tabular}{|p{5cm}|p{5cm}|}
		\hline 
		 \multicolumn{2}{|l|}{\textbf{Class Name:} File Management (Controller)} \\
		\hline
		\textbf{Responsibility:} & \textbf{Collaborators:} \\
		\hline
			Knows File Name & Chat Management Controller\\
			Knows File Size & File Database\\
			Knows File Type &\\
			Knows File Permissions &\\
			Handles coordination of sending, receiving, and storage of files &\\
			Handles the enforcement of file access permission &\\
		\vspace{0.1in} & \\
		\hline
		\end{tabular}
	\end{table}


	\begin{table}[ht]
		\centering
		\begin{tabular}{|p{5cm}|p{5cm}|}
		\hline 
		 \multicolumn{2}{|l|}{\textbf{Class Name:} File Database (Entity)} \\
		\hline
		\textbf{Responsibility:} & \textbf{Collaborators:} \\
		\hline
			Knows File Management & File Management \\
			Knows File Name &\\
			Knows File Size &\\
			Knows File Type &\\
			Knows File Permissions &\\
			Knows File Owner &\\
			Handles the storage of file data &\\
		\vspace{0.1in} & \\
		\hline
		\end{tabular}
	\end{table}


	\begin{table}[ht]
		\centering
		\begin{tabular}{|p{5cm}|p{5cm}|}
		\hline 
		 \multicolumn{2}{|l|}{\textbf{Class Name:} File Search (Boundary)} \\
		\hline
		\textbf{Responsibility:} & \textbf{Collaborators:} \\
		\hline
			Knows File Management & File Management\\
			Knows File Search Criteria (e.g., File Name, File Type, File Owner) &\\
			Handles file search functionality &\\
			Handles utilization of search criteria to locate and retrieve files matching specified conditions &\\
			Handles click-event of “Search” button &\\
		\vspace{0.1in} & \\
		\hline
		\end{tabular}
	\end{table}

	\begin{table}[ht]
		\centering
		\begin{tabular}{|p{5cm}|p{5cm}|}
		\hline 
		 \multicolumn{2}{|l|}{\textbf{Class Name:} File Retrieval (Boundary)} \\
		\hline
		\textbf{Responsibility:} & \textbf{Collaborators:} \\
		\hline
			Knows File Management & File Management \\
			Knows File ID & File Search \\
			Knows File Name &\\
			Knows File Type &\\
			Knows File Owner &\\
			Handles retrieval of files from storage based on specified file search criteria &\\
		\vspace{0.1in} & \\
		\hline
		\end{tabular}
	\end{table}

	\begin{table}[ht]
		\centering
		\begin{tabular}{|p{5cm}|p{5cm}|}
		\hline 
		 \multicolumn{2}{|l|}{\textbf{Class Name:} File Error (Boundary)} \\
		\hline
		\textbf{Responsibility:} & \textbf{Collaborators:} \\
		\hline
			Knows File Management & File Management \\
			Knows Error Type &\\
			Knows Error Details &\\
			Handles errors related to file operations (e.g., Invalid File Type Error, File Size Too Big) &\\
			Handles communication of error information to the user &\\
		\vspace{0.1in} & \\
		\hline
		\end{tabular}
	\end{table}

	\begin{table}[ht]
		\centering
		\begin{tabular}{|p{5cm}|p{5cm}|}
		\hline 
		 \multicolumn{2}{|l|}{\textbf{Class Name:}} \\
		\hline
		\textbf{Responsibility:} & \textbf{Collaborators:} \\
		\hline
		\vspace{1in} & \\
		\hline
		\end{tabular}
	\end{table}
\end{itemize}



% End Section
\clearpage

\appendix
\section{Division of Labour}
\label{sec:division_of_labour}
% Begin Section
Include a Division of Labour sheet which indicates the contributions of each team member. This sheet must be signed by all team members.
% End Section


\end{document}
%------------------------------------------------------------------------------
